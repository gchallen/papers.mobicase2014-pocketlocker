% 20 Apr 2014 : GWA : 0.5 page.

\section{Conclusion}
\label{sec-conclusion}

PocketLocker addresses an emerging need of mobile systems by crafting a
personal storage cloud from multiple personal devices.  While mobile devices
remain storage constrained, consumers today typically have copious unused
storage available in the form of desktop computers and external hard drives.
Previous cloud-based solutions have not been designed to target this space.
PocketLocker does, in a manner sensitive to both the reliability limitations of
consumer equipment and the energy limitations and communication costs of mobile
devices. It targets rarely changing files such as photos and videos, a typical
product of mobile devices.  The system distributes these files automatically
and redundantly amongst the devices of a user's personal cloud. The storage
devices themselves need not be homogeneous in size or type.  An intelligent
orchestrator arranges storage in a manner that maximizes usage of these
differing devices, maximizes redundancy, and minimizes network costs.
PocketLocker is free and uses no additional devices.  System storage and backup
policies are tuned based upon data gleaned from an extended testing period
using 100 smartphones.

Looking ahead, we envision incorporating energy minimization in the system's
chunk placement algorithm---presently, it considers network connectivity and
storage size.  The timing of when new files are uploaded to the PSC presently
hinges on fixed deadlines.  We would like to use a more flexible schedule that
considers network and energy conditions dynamically.

\section*{Acknowledgments}

The authors would like to thank the anonymous MobiCASE reviewers for their
constructive feedback. They are also grateful to the support staff of the
\PhoneLab{} team at the University at Buffalo for assistance with deploying the
platform changes needed to collect detailed file usage information. \PhoneLab{}
administrator Maulik Dave was particularly helpful in setting up our
experiments and providing information about the testbed.


