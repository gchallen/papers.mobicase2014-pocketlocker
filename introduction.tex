% 20 Apr 2014 : GWA : 1 page.

\section{Introduction}
\label{sec-introduction}

As smartphones become ubiquitous, viewers naturally expect to have access to
all of their personal content from these devices---to view all of
their photos and videos; play any track from their music collection; and browse
through all previously sent chats, texts, and emails. These
use cases require smartphones to efficiently access far more data than they
can store locally, and yet both existing cloud storage
solutions~\cite{dropbox, googledrive} and recent research filesystem
designs~\cite{mashtizadeh2013replication, peek2006ensemblue} require each
client store a complete replica. As users assemble multilple
devices---including smartphones, tablets, laptops, and desktops---that
collectively contain large amounts of storage, their storage capacity should
not be bottlenecked by the most storage-constrained device. Given the cost
and energy usage of flash storage we do not anticipate mobile device
storage capacity to keep pace with that of other types of devices.

To address this personal storage bottleneck we propose to allow users to
apply the same techniques used to build reliable cloud storage to create
\textit{personal storage clouds}. By combining available space on existing
personal devices, personal storage clouds can achieve a capacity far greater
than offered by free cloud storage tiers. By utilizing nearby personal
devices, personal storage clouds can provide better performance than cloud
storage. And by applying modern approaches to reliability and availability,
personal storage clouds can cope with failures and disconnections inherent to
personal devices.

We present the design and implementation of PocketLocker, a system enabling
scalable, reliable, and performant personal storage clouds (PSC). PocketLocker
is designed to store rarely-changed files, such as photos, music, and videos,
and to provide access to an entire personal storage cloud from any client
device. PocketLocker exploits the locality of devices within the PSC to arrange
rapid transfers over LANs when possible, and includes energy-saving features to
reduce battery drain on mobile clients. While PocketLocker uses direct
interaction between clients, it does not address the difficulties of building a
peer-to-peer distributed storage system. Instead, a cloud service called the
\textit{orchestrator} is used to maintain a consistent namespace and ensure
that backup and availability requirements are met.  Clients apply local data
caching policies that reflect their usage patterns and their interaction with
other clients. PocketLocker simplifies locating data within the personal
storage cloud by utilizing Reed-Solomon erasure
coding~\cite{reed1960polynomial}, allowing clients to reconstruct files as long
as they can acquire any set of mutually-redundant chunks.

Our paper makes the following contributions. First, we examine one month of
data from 100~users to better understand smartphone file access patterns. We
conclude that today's users are generating and accessing far more content
than can be stored directly on their mobile device, making file systems which
require each client to store a complete replica unusable. However, a survey
that we distributed to 47~people indicates that users do have free storage on
other personal devices. These results motivate PocketLocker's design.

Second, we present the design of PocketLocker and describe how it uses
multiple personal devices to build scalable, performant, and energy-efficient
personal storage clouds to store rarely-changed files such as music, videos,
and photos. PocketLocker uses erasure coding to divide files into multiple
chunks which are distributed across users' participating devices---which
can include smartphones, tablets, laptops, desktops, and dedicated storage
appliances. PocketLocker’s orchestrator, which runs as a cloud service,
distributes chunks across the users’ devices to maintain file
availability, maximize performance, and meet configurable backup
requirements.

Finally, we perform a detailed evaluation of PocketLocker that proceeds along
two lines. First, we utilize our file access traces to examine the impact of
several key PocketLocker design parameters and estimate the performance of
file access on PocketLocker. Second, we measure the energy consumption and
performance of an Android prototype as it accesses files under a variety of
real-world conditions. Our results confirm that by locating files
intelligently, PocketLocker can provide mobile users with energy-efficient
low-latency access to far more content than their mobile device can store
locally.

Our paper is structured as follows. Section~\ref{sec-motivation} presents
several results that motivate and inform PocketLocker's design, which we
present in detail in Section~\ref{sec-design}.
Section~\ref{sec-implementation} describes the implementation of our
current PocketLocker prototype for Android smartphones, which we evaluate in
Section~\ref{sec-evaluation}. Section~\ref{sec-related} compares PocketLocker
to similar systems before Section~\ref{sec-conclusion} discusses future work
and concludes the paper.
