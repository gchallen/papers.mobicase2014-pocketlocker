% 20 Apr 2014 : GWA : 1 page.

\section{Related Work}
\label{sec-related}

Mobile devices, being relatively new, did not contribute to the design of
prototype distributed file systems.  Early systems such as Coda~\cite{kistler1992disconnected} and
Ficus~\cite{guy1990implementation} were concerned with addressing the base problem of file caching
and replication.  The limitations of mobile devices, particularly constrained
storage and energy and intermittent connectivity, were not relevant.  Standard
network file systems such as NFS~\cite{nowicki1989nfs} did not provide direct
offline access or redundancy.

By contrast, there are robust commercial solutions
such as TimeMachine~\cite{timemachine} that furnish redundant storage from any device.
These cloud solutions are also typically limited in space and use third party storage.


The approach taken by
EnsemBlue~\cite{peek2006ensemblue} focuses on replicating files among mobile
devices.  Users can specify file groups that are automatically replicated.  
Cimbiosys~\cite{ramasubramanian2009cimbiosys} narrows this approach, implementing data filters such as file
type to determine replication policy.  Files that do not match
the filter are not replicated.  These approaches
limit access to files that can fit on a particular user's
device.  Additionally, since a file will not always be replicated,
there is no specific attempt to provide file
backup.  Since offline edits are allowed, conflicts occur and must be resolved.
PRACTI~\cite{belaramani2006practi} focuses on maximizing the tradeoffs of the general
goals of consistency, replication and independence.  This necessarily
unfocuses the specific needs of mobile storage.

PocketLocker aims to make all files in the PSC available.
Which files are maintained locally are determined by usage patterns and
network conditions.  Those that are not are still available with a possible
delay.  The size of the PSC can thus greatly exceed the local storage of
a particular device.  The chunk distribution system of PocketLocker
minimizes the impact of device failure and ensures file redundancy.

The Eyo system~\cite{strauss2010device} provides a distributed unified
namespace.  While file metadata is automatically replicated, replication of
file data is left to rules specified by client
programs.  Thus, files may not be replicated against failure.
If a user wants to access a nonlocal file,
the system can furnish its current location but does not
automatically retrieve it.  Editing a file offline can result in a conflict
that must be resolved.  The system addresses
storage pressure by pruning file version history without respect to possible
loss of redundancy.

The concept of separating the distribution of file metadata from data
underpins another system, Ori~\cite{mashtizadeh2013replication}.  Accessing
remote file data depends on being able to access that device directly.
Otherwise, the call fails.  Ori permits users to move versioned file histories
among devices---permitting offline editing but incurring storage overhead and
producing conflicts.  File backup
focuses on versioning.  Whether a file is replicated
depends upon whether the user has mounted a remote system.
Implementation of deliberate redundancy, in the form of multiple copies on
multiple devices, remains a function of user choices.  

PocketLocker handles replication of both
file metadata and data directly.  The system, having a bird's eye view of all
storage devices, can ensure that files are always chunked and replicated to
disparate devices to guard against failure.  Distribution of the chunks is
tuned to the differing storage capacities of different devices.  Storage
reclamation policy follows file history and usage patterns in order to
maximize backup potential.  The centralized design of PocketLocker also
allows it to handle potential remote access issues.  If a file or chunks are
not directly reachable from a client device due to firewall issues, the
Orchestrator can often mediate an indirect relay transfer rather than simply
failing.

